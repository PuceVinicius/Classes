\documentclass[a4paper, 11pt]{article}
\usepackage[brazil]{babel}
\usepackage[utf8]{inputenc}
\usepackage{amsfonts,amsmath,amssymb}
\usepackage{enumerate}
\usepackage[standard,thmmarks,thref]{ntheorem}
\usepackage[colorlinks=true,citecolor=blue]{hyperref}

\usepackage{xstring}

%\usepackage{mathtools}
%\usepackage{graphicx}

\title{MAC0239: Exercício-Programa 2 \\ \small{Lógica de Primeira Ordem}}
\author{Marcelo Finger}
\date{\today}


%% Redefinições (mfinger)
\renewcommand{\phi}{\varphi}
\renewcommand{\emptyset}{\varnothing}

%% Definições
\newcommand{\tuple}[1]{\langle{#1}\rangle}
\newtheorem{exemplo}{Exemplo}[section]
\newtheorem{lema}[exemplo]{Lema}
\newtheorem{teorema}[exemplo]{Teorema}
\newtheorem{questao}[exemplo]{Questão}

\newenvironment{quest}[1][@]{\IfStrEq{#1}{@}{\begin{questao}}{\begin{questao}[#1]}\rm}{\end{questao}}


\begin{document}
    \maketitle

\section*{Aviso Importante}

\begin{quote}\bf\Large
  Este  exercício deve  ser  criado e  processado por  \LaTeX\  e entregue  no
  formato PDF via paca.
\end{quote}

Não precisa entregar ``os fontes'', ou seja, o arquivo \texttt{.tex}.

Para facilitar  a vida de  você, estou  fornecendo os fontes  deste enunciado,
assim vocês já  tem um arquivo \texttt{.tex} para começar  a raquear (\emph{to
  hack}).

\section{Introdução}

O objetivo deste EP é desenvolver uma  série de fórmulas de lógica de primeira
ordem a partir  das quais um resultado de (falta  de) expressividade da lógica
de primeira ordem.  Ou seja, este EP será um ``estudo dirigido'' para provar o
seguinte resultado sobre a Lógica de Primeira Ordem (LPO).

\begin{teorema}[Resultado Principal do EP2] \label{teo:princ}
  Não existe  na Lógica de Primeira  Ordem uma fórmula que  seja verdadeira em
  todos os modelos com domínio finito e par, e apenas nestes. \qed  
\end{teorema}

\section{Questões}

Apresentar num  documento escrito e processado  em \LaTeX, a resposta  para as
seguintes questões.

\begin{quest}[Aquecimento] Apresentar  uma fórmula da LPO  que seja verdadeira
  em todos os  modelos $\mathcal{M}=\tuple{\mathcal{A}, \cdot^\mathcal{M}}$ em
  que o  domínio $\mathcal{A}$ é  infinito e não  é verdadeira em  modelos com
  domínio finito..

  Dica: use  uma assinatura  com apenas uma  constante, um  símbolo funcional
  unário e apenas o predicado da igualdade.

  Nota:  eu  chamei  isso de  aquecimento  pois  não  tem  nada a  ver  com  o
  Teorema~\ref{teo:princ}.  Ou tem?
\end{quest}

\begin{quest}
  Apresentar uma fórmula que:
  \begin{enumerate}[(a)]
  \item Seja verdadeira se e somente se (sse) o modelo tiver dois elementos
    
  \item Seja verdadeira sse o modelo tiver 4 elementos
  \end{enumerate}
\end{quest}

\begin{quest}\label{q:par}
  Apresentar uma fórmula que seja verdadeira sse, dado $n \in \mathbb{N}^+$, o
  modelo tiver $2n$ elementos.

  Dica: usar os conectivos generaizados:
  \[\bigwedge_{i=1}^{n} \phi_i = \phi_1 \land \ldots \land \phi_n \qquad
    \bigvee_{i=1}^{n} \phi_i = \phi_1 \lor \ldots \lor \phi_n\]
\end{quest}


\begin{quest}\label{q:lema}
  Dado $n  \in \mathbb{N}^+$,  apresentar uma fórmula  que seja  verdadeira em
  algum modelo com um número par de elementos até $2n$.

  Dica: a mesma da Questão~\ref{q:par}.  
\end{quest}


\begin{quest}[Finalmente] Provar o Teorema~\ref{teo:princ}.

  Dica: Usar um argumento de compacidade, como o feito em sala de aula, e usar
  as fórmulas apresentadas na Questão~\ref{q:lema}.  
\end{quest}
\end{document}

%%% Local Variables:
%%% mode: latex
%%% TeX-master: t
%%% End:
